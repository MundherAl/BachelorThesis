\subsection{Future Work}

Topic modeling and sentiment analysis are an exciting new approach to text synthesis. While the thesis has shown that no clear information was extracted from the data, there are many methods to deliver different or better results.

LDA is considered to be a state-of-the-art model for topic modeling. But it comes with its own limitations that can perhaps cause to be harmful for data sets such as the one used in this project. Applying other models such as NMF could be an alternative that produces more meaningful results.

The project used a simple lexicon-based approach to sentiment analysis, which as discussed, has its fair share of shortcomings. A new exciting approach to sentiment analysis is a combination of using topic modeling to extract topics by sentiment, so-called joint sentiment/topic models.~\cite{lin2009joint} This could be a more suitable alternative to investigate how victims feel by topic and might deliver far more meaningful results. It might be possible to extract topics linked to high negative sentiment and link them to certain feelings, such as shame or fear.

\subsubsection*{Further Research on Coping Mechanisms}

It has been shown that there is a lack of research on the psychological impact of cybercrime, let alone the coping mechanisms that victims use to recover from their experience. The research community and victims alike could benefit from more insights into the coping mechanisms that victims use to recover from their experience. Reproducing the research results, for example those of Jansen and Leukfeldt, could be a valuable contribution~\cite{jansen2018coping} and as conducting surveys.

\subsubsection*{Data Scarcity}

Data is very scarce in this field and there need to be more efforts towards data collection and organization. The more data there is, the more research can be done. Machine-learning based sentiment analysis could be a valuable tool to analyze the data and extract meaningful information, but this was not possible due to the lack of data, as well as labels for the data. A framework for cultivating data can increase the amount of research in the field of cybercrime victimization.


\subsubsection*{A Social Platform for Supporting Cybercrime Victims}

The context of Scamalert could be a major factor to skewing the data towards neutral sentiment. Users seem to be inclined to report their encounter rather than share their experience as a whole. A platform that is more oriented on supporting victims and helping them recover from their experience could be a better source of data. Creating an environment for victims to share and support each other could provide a valuable service to the victims of cybercrime. Jansen and Leukfeldt's research indicates that the support of others is an effective coping mechanism for victims of cybercrime.~\cite{jansen2018coping}

Such a platform could also be used to improve the data ecosystem on this subject. Victims could be prompted to label their stories with the feelings they experienced, which could be used to train a sentiment analysis model. A built-in API could be used to extract the data and make it available for research purposes.