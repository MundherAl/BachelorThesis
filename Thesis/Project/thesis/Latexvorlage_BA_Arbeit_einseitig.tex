\documentclass[12pt,english,titlepage,a4paper]{article}

\usepackage{babel}

\begin{document}

%===========================================================
\begin{titlepage}
\begin{center}

\textbf{\LARGE How Victims Cope with Cybercrime Incidents: A Topic Modeling Approach}

\bigskip\bigskip
\textbf{Bachelor Thesis}

\bigskip
\textbf{Mundher A. Y. Al-Ahmadi}



\vfill
Chair of Digital Innovation and Entrepreneurship\\ 
Faculty of Mathematics and Natural Sciences \\ 
Heinrich Heine University D\"usseldorf

\bigskip
\today

\bigskip
Supervisor: Prof. Dr. Steffi Haag \\
Second supervisor: Prof.\ Dr.\ Melanie Schmidt

\end{center}
\end{titlepage}

\thispagestyle{empty}\mbox{}\pagebreak
\setcounter{page}{0}

%===========================================================
% Abstract
%===========================================================
\section*{Abstract}
\addcontentsline{toc}{section}{Abstract}
Lorem, ipsum dolor sit amet consectetur adipisicing elit. Facere corporis reiciendis accusantium adipisci veritatis facilis earum nemo debitis, minima eius eum vitae, rerum eaque excepturi molestias quibusdam quo? Quibusdam, obcaecati.
Error quis eaque ipsam, veritatis hic nihil odit, perspiciatis maxime dolor animi optio sunt molestiae? Magnam maxime dolores nobis labore minus, tempore iste nulla voluptatibus eum tenetur commodi natus doloremque!
Eum beatae ut provident corporis pariatur corrupti possimus recusandae officiis adipisci dolores tenetur accusamus ipsa sequi tempore numquam autem, laborum commodi quis minima voluptatibus nam unde eveniet sint animi. Quis?
Consectetur, reprehenderit quisquam. Omnis facere eveniet dolore exercitationem quo ducimus consectetur quis facilis consequuntur, cupiditate architecto, reiciendis molestiae officia quaerat maiores optio, labore sapiente doloribus ex eaque recusandae corrupti iure?
Ipsum architecto eveniet possimus rem quisquam, consequuntur consequatur totam magnam eum aspernatur? Quo, aliquid tempora ducimus perspiciatis inventore saepe, provident ea molestiae quos eius quia et? Assumenda voluptatum voluptates itaque?
\pagebreak



%===========================================================
% Table of Contents
%===========================================================

\tableofcontents
\pagebreak


%===========================================================
% Introduction
%===========================================================
\section{Introduction}
\addcontentsline{toc}{section}{Introduction}

As the number of internet users increases, the number of cybercrime incidents increases as well. Unsurprisingly, the COVID-19 outbreak, which acted as a catalysor for digitalization, has led to a surge in cybercrime incidents.~\cite{Monteith2021Increasing} There is no doubt about cybercrime being a significant problem. An estimation by Cybersecurity Ventures suggests that cybercrime will cost the world economy around 8 trillion USD in 2023. This figure is projected to continue growing in the coming years.~\cite{cybersecurity-ventures-cybercrime-report}

While carrying such a significant financial cost, cybercrime also has a psychological impact on victims. Victims of cybercrime report feelings of shame, embarrassment, shock, and even trauma.~\cite{jansen2018coping} Ho and Luong have identified a research gap in cybecrime victimization literature, which is an indicator of the lack of understanding of the seemingly adverse effects of cybercrime on victims. Ho and Luong's results seem to indicate that there is no substantial body of research on coping, based on their keyword analysis.~\cite{horesearch}

What is also a contributing factor to the research gap is the lack of a unified definition of cybercrime. There have been several attempts to define cybercrime, but also criticism of searching for such a definition. Gordon and Ford believe there are benefits to deleting the term "cybercrime" from the lexicon entirely, as they are too broad to be defined.~\cite{gordon2006definition} Borwell, Jansen and Stol state that there is a lack of theoretical frameworks that explain the impact of cybercrime on victims and argue that the Shattered Assumptions Theory (SAT), a victomological theory that provides a theoretical perspective for understanding psychological responses~\cite{janoff1983theoretical}, seems to be a suitable framework to explain victimization in cybercrime.~\cite{borwell2022psychological} Such theories contribute to coping with crime and they may well be applicable to cybercrime as well.

All of this points to the need for more research on the subject. Fortunately, there seems to be a growing interest in the topic.~\cite{horesearch} One way to further stimulate research on the topic is to provide a comprehensive overview of the current state of research. This can be achieved by conducting a literature review. There are many types of literature reviews. Grant and Booth have analysed 14 types of literature reviews and their outputs. Literature reviews serve as an important tool for researchers to find out what is known about a subject, what remains unknown, recommendations for practice and for future research.~\cite{grant2009typology}



\pagebreak


%===========================================================
\section{Methodology}

%===========================================================
\subsection{Data Collection}

%===========================================================
\subsection{Data Preprocessing}

%===========================================================
\subsection{Latent Dirichlet Allocation}

%===========================================================
\subsection{Parameter Tuning}



%===========================================================
% References
%===========================================================
\pagebreak
\bibliographystyle{plain}
\bibliography{bibliography.bib}



%===========================================================
\pagebreak\noindent
\textbf{\LARGE Erkl\"arung}
\addcontentsline{toc}{section}{Declaration}

\bigskip\bigskip
\noindent 
I hereby confirm that I have independently written the 
bachelor's thesis and have not used any sources or aids 
other than those specified.
\bigskip
\noindent
% Date of Declaration

\bigskip\bigskip\bigskip
\noindent
% Automatic date
D\"usseldorf, \today \\
(Mundher Al-Ahmadi)

% bibliography
% \bibliographystyle{plain}



\end{document}