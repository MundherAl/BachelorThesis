\subsection{Data Analysis}

A total of 3489 stories will be analyzed in two parts: a topic model using Latent Dirichlet Allocation (LDA) and a sentiment analysis using the VADER sentiment analysis tool. The topic model will be used to identify the most common topics in the stories and the sentiment analysis will be used to identify the sentiment of the stories.

\subsection{Topic Modelling}

As mentioned in the introduction, topic modeling is a text-mining method to identify and classify data. More specifically, topic modeling is used to identify topics in a corpus of documents and classifying a distribution of words to them. A simple example would be that a sentence containing the words "politician bill government president economic growth" could be attributed to a topic "politics". We can also see that the words "economic" and "growth" could also be attributed to another topic "economy". Topic modeling extrapolates this idea to a corpus of documents and identifies the most common topics in the corpus. There are many methods to perform topic modeling, all with their own advantages and disadvantages.

\subsubsection{Latent Dirichlet Allocation (LDA)}

