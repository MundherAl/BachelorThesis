\documentclass[12pt]{article}
\usepackage{graphicx} % Required for inserting images
\usepackage{natbib} % Required for bibliography

\title{Research Proposal - Cybersecurity and Crime}
\author{Mundher Al-Ahmadi}
\date{\today}
\begin{document}

\maketitle

\section{Introduction}

The number of internet users has been steadily on the rise and projected to
continue to rise. As digitalization expands, particularly since the 
outbreak of COVID-19, instances of cybercrime have similarly been 
increasing~\cite{Monteith2021Increasing}, despite being significantly underreported.~\cite{IOCTA2021} 
This raises many questions about cybercrime and security. Among those questions is
the effect of cybercrimes such as phising and identity theft on victims.

\section{Background and Significance}

\subsection{Costs of Cybercrime}

The amount of internet users world-wide has increased from about 1 billion
to 5.3 billion in the period between 2005 and 2022.~\cite{StatistaInternetUsers}
This figure is expected to continue growing, as well as cybercrime~\cite{rise-of-cybercrime},
although it is severely underreported. Cybercrime has very significant
financial costs that are also projected to increase at a high rate. 
According to the Europol's Internet Organized Crime Threat Assessment (IOCTA)~\cite{IOCTA2021}, 
cybercrime is becoming more aggressive and confrontational.
Such crimes include but are not limited to cyberstalking, sexual extortion
as well as child abuse. The effects of such crimes can be very severe.~\cite{IOCTA2021}

Many cybercrimes are carried out at a mass scale with very low costs. For example,
phising attacks can be carried out with very little cost. Another example are 
DDoS attacks. DDoS attacks are performed by bots, are accessible to any individual,
and can be carried out for as little as 10 USD.~\cite{gomez2020dark}
There are many estimations on the financial costs of such attacks on individuals and companies,
and they seem to be significant. Cybercrimes are estimated to cost the world economy
about 8 trillion USD in 2023.~\cite{cybersecurity-ventures-cybercrime-report}

The psychological effects of such crimes are also relevant. Victims report feelings
of that range from shame and embarrassment to shock and even trauma.
Jansen and Leukfeldt report some coping mechanisms that phishing victims use. The first 
coping mechanism was reporting the crime to their bank. Victims also change their online
environment by more frequently installing software security updates. While these coping
strategies are problem-focused, they could instill a feeling of security and confidence
when using online banking. Another coping mechanism is emotion-focused, which is to
tell friends and family about the incident. Though, a victim reports that their feelings
about the incident prevented them from telling anyone.~\cite{jansen2018coping}

\subsection{Problem scaling}


The growth of internet usage and the rate of cybercrime is a cause for great
concern for a number of reasons:

\begin{enumerate}
    \item a 6\% compound annual growth rate in internet users introduces
    billions of new potential victims.~\cite{StatistaInternetUsers}
    \item Limited levels of awareness against cybercrime prevention.~\cite{enisa-raising-awareness}
    \item Unknown levels of awareness about the psychological and financial
    damages to victims.
\end{enumerate}

The sum of these reasons implies a highly neglected cause. 
However, there is reason to believe that there is much potential 
for things to improve. The number of unfilled cybersecurity jobs has grown
from one million to 3.5 million between 2013-2021.~\cite{cybersecurity-ventures-cybercrime-report}
McKinsey believes that there may be a big addressable market for cybersecurity. As of 2021,
The total addressable market is predicted to be valued between 1.5 trillion to 2 trillion dollars, which
is about 10 times as big as the vended market.~\cite{McKinsey2022Survey} 
There has also been developments in rethinking cybersecurity models and frameworks with
companies like RSA focusing on identity authentication and authorization.
Although this might mean a great deal of potential for innovations in 
prevention, it has no implications for financial and psychological 
recovery for victims.

\section{Research Methods}

A keyword search analysis will be performed as a means to measure the amount of 
research on the topic of cybercrime victmization. Furthermore, literature on the topic
will be examined and a topical analysis using Latent Dirichlet Allocation (LDA) will be 
performed on text data from existing literature, as well as on posts from online forums
and social media platforms. Using LDA, the topics of discussion will be identified.

\section{Expected Outcomes}

The analysis will provide a topic model of the discussion on cybercrime victimization.
The model can be used to identify the most discussed topics, as well as the most
neglected topics. This facilitates literature reviews and supports hypothesis generation.

\bibliographystyle{plain}
\bibliography{references}

\end{document}
